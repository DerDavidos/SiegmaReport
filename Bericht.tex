\documentclass[
  pdftex,
  chapterprefix,
  headsepline,
  footsepline,
  colordvi,
  11pt,
  a4paper,
  halfparskip,
  final,
  appendixprefix,
  bibtotoc]{scrbook}
% uncomment the following line (mutual exclusive to the one above) to enter the draft mode.
%\documentclass[colordvi,11pt,a4paper,halfparskip,draft,appendixprefix,bibtotoc]{scrbook}

%
% neues if definieren, um zwischen PDF und DVI entscheiden zu k�nnen.
%
\usepackage{ifpdf}
\ifx\pdfoutput\undefined
\pdffalse %not PDFLaTeX
\else
\pdfoutput=1
\pdftrue
\fi
%\tracingstats=2
%\usepackage{layout}
% german language support (hyphenation etc)  
\usepackage[english]{babel}

% for prettier tables
\usepackage{booktabs}

% support for latin1 characters. That means you can enter umlauts directly
% no need for "a "u "o "s anymore
\usepackage[utf8x]{inputenc}
%\usepackage[latin1]{inputenc}

% provides the \url{} command to pretty print urls
\usepackage{url}

% needed for a german bibliography-style (s. below)
\usepackage{bibgerm}

% allows text flowing around figures.
\usepackage{wrapfig}

% allows to \includegraphics
\usepackage{graphicx}

% defines some standard colornames like "black" etc.
\usepackage{color}

% allows to color tablecells
\usepackage{colortbl}

% provides an easier interface to if-then-else constructs in 
% custom macros
\usepackage{ifthen}

% allows tables to break over pages.
\usepackage{supertabular}

% allows to have different kinds paper orientations in the same pdf-documnent
\usepackage{pdflscape}

% allows to specify absolute texpos for textboxes. This is generally only important for the titlepage
\usepackage[absolute]{textpos}

% allows to enumerate different figures with a) b) in the same figure-environment.
\usepackage{subfigure}

% finetune the gaps between figure and text in the subfigure environment (basically close the gap as much as possible)
\renewcommand{\subfigtopskip}{0pt}
\renewcommand{\subfigbottomskip}{0pt}

% some color definitions for the pdf statements below
\definecolor{mygrey}{rgb}{0.45,0.45,0.45}
\definecolor{mydarkgrey}{rgb}{0.2,0.2,0.2}
\definecolor{red}{rgb}{1.0,0.33,0.33}
\definecolor{orange}{rgb}{1.00,0.73,0.33}
\definecolor{yellow}{rgb}{0.95,0.92,0.}
\definecolor{lightgreen}{rgb}{0.3,0.95,0.46}
\definecolor{titleblue}{rgb}{0.03,0.10,0.46}

\ifpdf
% Metadata and configuration of the pdf output:
% Do not forget to enter the correct title, author, subject und keywords

% For screen viewing it is nice to have references marked in a slightly different
% color than the rest of the text. Since they will be hyperlinks to the 
% referenced objects.
\usepackage[pdftex,
             pdftitle={},
             colorlinks,
             linkcolor={mydarkgrey},
             citecolor={mygrey},
             urlcolor={black},
             plainpages={false},
             bookmarksnumbered={true},
             pdfauthor={},
             pdfsubject={},
             pdfkeywords={},
             pdfstartview={FitBH}]{hyperref}

% For the final printouts (remember - you need at least three - one for each examiner and one for the archive 
% [ This might have changed - so contact the "Pr�fungsamt" about the current regulations !! ] - it is better
% to have all text in the same color (namely black).
% 
%\usepackage[pdftex,
%            pdftitle={},
%            colorlinks,
%            linkcolor={black},
%            citecolor={black},
%            urlcolor={black},
%            plainpages={false},
%            bookmarksnumbered={true},
%            pdfauthor={},
%            pdfsubject={},
%            pdfkeywords={},
%            pdfstartview={FitBH}]{hyperref}
\pdfcompresslevel=9
\fi

% some configuration for the amount of text on a single page
\usepackage{typearea}
\areaset[1.5cm]{418pt}{658pt}
\setlength{\headheight}{37pt}

% Enter author and title for the titlepage.
\author{}
\title{}

% To avoid nasty mistakes like having comments directly in the textflow
% the following \todo macro was defined. With that you can enter
% \todo{What I still have to do here} 
% inside of your text and a marker will appear at the page's margin with the 
% text "What I still have to do here".
% The first line activates this feature. If you comment it out and uncomment
% the second line below there will be no error messages and no todos will be shown
% anymore. So - even if you have forgotten to delete one of them - they will not appear
% in the final printout. 
\newcommand{\todo}[1]{\marginpar{\textcolor{red}{ToDo:} #1}}
%\newcommand{\todo}[1]{}

% We recommend to split your document into several files. Usually one for every chapter is a 
% good idea. If you follow this guideline (how to assemble these files in a single document
% see two paragraphs below) you will be able to single out chapters via the \includeonly{}
% command. Using this mechanism page numbering and references of the full run before will be
% preserved. This also nice, if your latex run tends to get slow and you need to fine tune 
% some formatting in one chapter - just include that one. The rest (or at least the ones before
% the one currently under construction) will remain untouched. This means a boost in compilation time.
%\includeonly{chapter2}

\begin{document}
% the next two lines influence the detailedness of the table of contents
% and to what structure depth numbers are written before sections/subsections/paragraphs
% You should not touch this
\setcounter{tocdepth}{2}
\setcounter{secnumdepth}{3}
\frontmatter
% here the titlepage is included. Look into the file "titlePage.tex" to 
% adapt it to your needs (name, title etc.)
%!TEX root=../../Vorname_Nachname_Diplomarbeit.tex

% Titelseite braucht folgenden  Eintrag
% \usepackage[absolute]{textpos}
% textpos ist nicht Bestandteil von tetex
% kann aber von dante heruntergeladen werden
\begin{titlepage}
\vspace*{-1cm}
\newlength{\links}
\setlength{\links}{0.9cm}
\setlength{\TPHorizModule}{1cm}
\setlength{\TPVertModule}{1cm}
\textblockorigin{0pt}{0pt}

\sf
\LARGE

\begin{textblock}{16.5}(2.8,2.6)
 \hspace*{-0.25cm} \textbf{UNIVERSITÄT DUISBURG-ESSEN} \\
 \hspace*{-1.15cm} \rule{5mm}{5mm} \hspace*{0.05cm} FAKULTÄT FÜR INGENIEURWISSENSCHAFTEN\\
 \large{}ABTEILUNG INFORMATIK UND ANGEWANDTE KOGNITIONSWISSENSCHAFT\\
\end{textblock}


%Hier Titel, Name, und Matrikelnummer eintragen, \\ make a newline
\begin{textblock}{14.5}(3.2,8.5)
  \large
{ \bf Projektarbeit} \\[1cm]
{\LARGE \Large\bf Drink Mixing Machine} \\[1.3cm]
David Lewakis\\
Matrikelnummer: 3080397\\
\end{textblock}



\begin{textblock}{10}(10.5,17.5)
\includegraphics[scale=1.0]{content/images/unilogo.pdf}\\
\normalsize
\raggedleft
Eingebettete Systeme der Informatik, Abteilung Informatik \\
Fakultät für Ingenieurwissenschaften \\
Universität Duisburg-Essen \\[2ex]

\today\\[15ex]
\raggedright
% Supervisors
{\bf Erstgutachter:}  \\
{\bf Zweitgutachter:}\\
{\bf Zeitraum:} 12.Oktober 2022 - 01.Februar 2023\\
\end{textblock}

\end{titlepage}

\cleardoublepage

\section*{Abstract}




\cleardoublepage

\tableofcontents

%\listoffigures
\mainmatter

% To assemble the whole document
% Please be aware that each file will begin on a new page
% therefore chapters should be put into such a file.
% There cannot be an include statement inside of an "included" file.
% So if you want to further divide your document - use \input inside of 
% the included files. \input will not begin on a new page.
\chapter{Introduction}

\section{Motivation}
\section{Aufgabenstellung}
\section{Aufbau der Arbeit}

\chapter{Chapter 2}


% Appendix chapters to be put here. They will be enumerated with capital letters 
% if you  did not change the \documentclass options.
\begin{appendix}


%\include{appendix_chapterA}
\end{appendix}
%Ende Anhang

%Bibliography
% We strongly recommend to use bibtex to manage your bibliography. It helps you
% structure your references and helps avoiding missing important data for a correct
% quotation. If you have no other idea jabref (http://jabref.sourceforge.net/)
% might be a good idea (Jave runtime environment needed).
% This style is good to use in german master thesis'. You need to have activated
% \usepackage{bigerm} above.
% For english documents just use apalike.
\bibliographystyle{geralpha}

% to finally announce where your bibliography is stored use
\bibliography{content/references/references}
% it is also possible to have several files separated by comma. 
%Bibliographie Angaben mit \bibliography{}

%!TEX root=../../Vorname_Nachname_Diplomarbeit.tex
%pagenumbering{null}

\ 

%clearpage
\cleardoublepage

\ 


\pagestyle{empty}

\textbf{Versicherung an Eides Statt}\\

Ich versichere an Eides statt durch meine untenstehende Unterschrift,
\begin{itemize}
\item[-] dass ich die vorliegende Arbeit - mit Ausnahme der Anleitung durch die Betreuer - selbstständig ohne fremde Hilfe angefertigt habe und
\item[-] dass ich alle Stellen, die wörtlich oder annähernd wörtlich aus fremden Quellen entnommen sind, entsprechend als Zitate gekennzeichnet habe und
\item[-] dass ich ausschließlich die angegebenen Quellen (Literatur, Internetseiten, sonstige Hilfsmittel) verwendet habe und
\item[-] dass ich alle entsprechenden Angaben nach bestem Wissen und Gewissen vorgenommen habe, dass sie der Wahrheit entsprechen und dass ich nichts verschwiegen habe.
\end{itemize}
Mir ist bekannt, dass eine falsche Versicherung an Eides Statt nach \S 156 und \S 163 Abs. 1 des Strafgesetzbuches mit Freiheitsstrafe oder Geldstrafe bestraft wird.
\vfill
Duisburg, \today\\
$\overline{\parbox{4.8cm}{(Ort, Datum)}} ~~~~~~~~~~~~~~~~~~~~~~~~~~~ \overline{\parbox{7cm}{(David Lewakis)}}$


\end{document}
%%% Local Variables:
%%% mode: latex
%%% TeX-master: t
%%% End:
